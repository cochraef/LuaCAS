\documentclass{article}
\usepackage{luacas}
\usepackage{forest}
\usepackage{verbatim}
\usepackage{pgfplots}

\usetikzlibrary{calc}

\begin{document}



\begin{CAS}
    vars('x')
    disp(int(1/(x^3+1), x))
\end{CAS}

On the other hand:

\begin{CAS}
    disp(int(1/(x^2-x+1),x))
\end{CAS}

The irreducible factorizations of the polynomials $\sum_{i=0}^j x^i$ for $j$ from 0 to 10 are:  
\begin{CAS}
    vars('x')
    p = 0
    for i in range(0, 10) do
        p = p + x^i
        disp(factor(p))
    end
\end{CAS}

The polynomial is \get{p}.

\parseforest*{p}
\bracketset{action character = @}
\begin{forest}
    for tree={draw,font=\ttfamily}
    @\forestresult
\end{forest}

\begin{CAS}
    vars('x','y')
    a = x*y
    b = 3*x*y
    c = a+b
\end{CAS}
$\print{c}$

\begin{CAS}
    f = diff(x^2*y+x*y,{x,2},y)
\end{CAS}
\[ \print*{f} = \print{f} \] 
\vprint*{f}



\begin{CAS}
    vars('x')
    f = x^2+2*x-2
    g = x^2-1
    subs = {[x] = f}
    dh = substitute(subs,g)
\end{CAS}

\begin{CAS}
    h  = int(dh,x)
\end{CAS}
\[ \print{h} \] 

\begin{CAS}
    h = simplify(int(dh,x)+10)
\end{CAS}
\[ \print{h} \]

\begin{CAS}
    r = roots(dh) 
\end{CAS}

\luaexec{
    v = {}
    for i=1,4 do 
        table.insert(v,substitute({[x]=r[i]},h):simplify())
    end
}
\[ \print{r[1]}, \quad \print{r[2]}, \quad \print{r[3]}, \quad \print{r[4]} \] 
\[ \print{v[1]}, \quad \print{v[2]}, \quad \print{v[3]}, \quad \print{v[4]} \] 

%\yoink{h}
\yoink{dh}
\yoink{r[1]}[a]
\yoink{r[2]}[b]
\yoink{r[3]}[c]
\yoink{r[4]}[d]

\yoink{v[1]}[A]
\yoink{v[2]}[B]
\yoink{v[3]}[C]
\yoink{v[4]}[D]

\newcommand{\fetch}[1]{
    \directlua{tex.print(tostring(#1))}
}
\NewDocumentCommand{\store}{m O{#1}}{
    \expandafter\def\csname #2\endcsname{%
        \directlua{
            input = #1
            if not input[1] then
                tex.print{tostring(input)}
            else
                tex.print("{")
                for _,entry in ipairs(input) do 
                    tex.print(tostring(entry),",")
                end
                tex.print("}")
            end
        }%
    }%
}%

\store{h}
\store{r}
\store{v}
\pgfmathsetmacro\a{\r[3]}
$\a$

\begin{tikzpicture}
    \foreach \k in {0,1,2,3}{
        \pgfmathsetmacro\a{\r[\k]}
        \draw (\a,0) circle (\a);
    }
\end{tikzpicture}

\begin{tikzpicture}
    \begin{axis}[legend pos = north west]
        \addplot [domain=-3.5:1.5,samples=100] {\h};
        \addlegendentry{$f$};
        \addplot[densely dashed] 
            [domain=-3.25:1.25,samples=100] {\dh};
        \addlegendentry{$df/dx$};
        \addplot[gray,dashed,thick]
            [domain=-3.5:1.5] {0};
            \coordinate (O) at (axis cs:0,0);
            \coordinate (X) at (axis cs:1,0);
            \coordinate (Y) at (axis cs:0,1);
    \end{axis}
    \begin{scope}[x={($(X)-(O)$)}, y={($(Y)-(O)$)}, shift={(O)}]
        \foreach \k in {0,1,2,3}{
        \pgfmathsetmacro\a{\r[\k]}
        \pgfmathsetmacro\b{\v[\k]}
        \draw[fill=purple,purple]
            (\a,0) circle (1.5pt);
        \draw[fill=purple,purple]
            (\a,\b) circle (1.5pt);
        \draw[purple,dashed]
            (\a,0) -- (\a,\b);
        }
    \end{scope}
\end{tikzpicture}

\begin{CAS}
    vars('x','h')
    f = 2*x/(x^2+1)
    subs = {[x]=x+h}
    q = (substitute(subs,f)-f)/h
\end{CAS}
\[ \print{q} \]
\begin{CAS}
    q = expand(q)
\end{CAS}
\[ \print{q} \] 

\luaexec{
    den = {}
    num = {}
    if q.operation == BinaryOperation.ADD then 
        for _, expr in ipairs(q.expressions) do
            local numpart = Integer.one()
            if expr.operation == BinaryOperation.MUL then 
                for _,subexpr in ipairs(expr.expressions) do 
                    if subexpr.operation == BinaryOperation.POW and subexpr.expressions[2] == -Integer.one() then    
                        for _,term in ipairs(den) do 
                            if subexpr.expressions[1] == term then 
                                goto continue
                            end
                        end
                        table.insert(den,subexpr.expressions[1])
                        ::continue::
                    else 
                        numpart = numpart*subexpr 
                    end
                end
            end
            if expr.operation == BinaryOperation.POW and expr.expressions[2] == -Integer.one() then 
                for _,term in ipairs(den) do 
                    if expr.expressions[1] == term then 
                        goto continue
                    end
                end
                table.insert(den,subexpr.expressions)
                ::continue::
            end
            table.insert(num,numpart)
        end
    end
    denominator = Integer.one()
    numerator   = Integer.zero()
    for _,expr in ipairs(den) do 
        denominator = denominator*expr
    end
    denominator = denominator:simplify()
    for _,expr in ipairs(num) do 
        numerator = numerator + expr*denominator 
    end
    numerator = numerator:expand():factor()
    common = numerator/denominator
    common = simplify(common)
}
\[ \print{denominator} \] 
\[ \print{numerator} \] 
\[ \print{common} \] 


\parseforest{q}
\begin{forest}
    @\forestresult
\end{forest}



\end{document}