\documentclass{article}
\usepackage{luacas}

\usepackage{xcolor}

\begin{document}

Consider the expression: $x^g$. The derivative with respect to $x$ is:
\[ \diff{x^g+x} \] %this working?
while the derivative with respect to $g$ is:
\[ \diff[var=g]{x^g+x} \] 
Symbolic integration:
\[ \int \frac{x+5}{x^2+3x+2}\ \mathrm{d}x = \eval{int((x+5)/(x^2+3*x+2),x)} \]
Let's use \verb|\def\expr{(x+5)/(x^2+3*x+2)}| to save an expression into the macro \verb|\expr|. 
\def\expr{(x+5)/(x^2+3*x+2)}
\[ \eval{\expr} \] 
This might be helpful; we could write:
\[ \int \eval{\expr}\ \mathrm{d}x = \eval{int(\expr,x)}.\] 
Let's try some integer arithmetic:
\[ \frac{108}{37} + \frac{76}{17} = \eval{108/37+76/17}.\]
Let's try gcd:
\[ \gcd(108,76) = \luaexec{
local a = Integer(108)
local b = Integer(76)
tex.print(Integer.gcd(a,b))
} \] 
\[ f'(x) = \luaexec{A=parse('DD(x,x^2)')
    tex.print(A:tolatex())} \quad \Rightarrow \quad 
    f(x) = \luaexec{
        B=parse('int(2*x,x)')
        tex.print(B:tolatex())} \] 
\luaexec{tex.print("\\def\\A{",A:tolatex(),"}")}
What is A? It is $\A$.

Let's try:
\[ \luaexec{
    y=SymbolExpression("x")
    Q=DD(y,y)
    A=DD(SymbolExpression("x"),SymbolExpression("x^3/2"))
    B='DD(x,x^3/2)'
    C=parse(B)
    print("This is the value of A:", tostring(A)) 
    print("This is the value of B:", tostring(B)) 
    print("This is the value of C:", tostring(C)) 
    tex.print(tostring(A),"\\qquad")
    tex.print(tostring(B),"\\qquad")
    tex.print(tostring(C),"\\qquad")
    tex.print(C:tolatex(),"\\qquad")
    tex.print(tostring(parse('DD(y,y^4)')))
    } 
\] 
Pop back in?
\[ \luaexec{
    tex.print(C:tolatex())
} \] 
Use \verb|C| in eval?
\[ \eval{int(C,x)} \] 
What if we save \verb|C| into a \LaTeX\ macro: \luaexec{tex.print("\\def\\C{",tostring(C),"}")}
Now \verb|\C| should contain the contents of \verb|C| as a string. Let's try to use it:
\[ \eval{int(\C,x)} \] 
Woot!
\section{(Pseudo) Bugs}
{\color{gray} 
\[ \int \frac{1}{x^2+1}\ \mathrm{d}x = \eval{int(1/(x^2+1),x)}\]
Maybe get irreducible quadratics into the integration table.
}
\end{document}

ideally:
\luacas{
    A=x*sin(x)
    B=DD(x,A)
    C=int(B,x)
    display(C) %shortcut to tex.print(C:tolatex())
}