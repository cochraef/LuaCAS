\documentclass{article}
\usepackage{luacas}
\usepackage{parskip}
\usepackage{comment}
\usepackage{forest}
%\usepackage{tikz-qtree}
\usepackage{pgfplots}
\usepackage{asypictureB}
\pgfplotsset{compat=1.18}

\usepackage{lipsum}
\usepackage{microtype}

\begin{document}

\begin{CAS}
    vars('x','h')
    f = x^2 - (2*x+h)
\end{CAS}
\[ \print*{f} = \print{f} \] 



Hi!
\begin{CAS}
    vars("x","y")
    g = diff(1/(x^2+1),{x,2})
\end{CAS}%
We have: 
\[ \print*{g} = \print{g}.\] 

\vprint*{g}

\parseforest{g}
\bracketset{action character = @}
\begin{forest}
    for tree={draw,font=\ttfamily}
    @\forestresult
\end{forest}
\begin{tikzpicture}
\printshrub{g}
\end{tikzpicture}

Here is the evaluated tree for $\print{g}$:

\parseforest*{g}
\begin{forest}
    @\forestresult
\end{forest}

\luaexec{
    h = g:autosimplify()
}

\luaexec{
    tex.print("\\def\\a{",tostring(h),"}")
}
\begin{asypicture}{width=6cm}
    import settings;
    settings.tex = "xelatex";
    import graph;
    import contour;
    size(6cm,5cm,IgnoreAspect);

    real f(real x){
        return @a;
        }
    path q = graph(f,0,2,operator..);

    draw(q,blue);

    xaxis("$x$",LeftRight,LeftTicks);
    yaxis("$y$",BottomTop,RightTicks);
\end{asypicture}

\luaexec{
    vars("x","f","foo")
    a = f(x*x)
    b = foo(x*x)
}
$\print{DD(a,x)}$

$\print{DD(DD(b,x),x)}$


\begin{CAS}
    a = x^4 -1
    a = a:autosimplify()
    a = a:topolynomial()
    b = x^2+1
    b = b:topolynomial()
    c = a \% b
\end{CAS}
$\print{c}$

\begin{CAS}
    vars('x','y')
\end{CAS}

\begin{CAS}
    a = x+y
\end{CAS}


\def\a{x^2+x*y+y^2}
\begin{CAS}
    a = \a
\end{CAS}
$\print{a}$ vs $\a$

\begin{CAS}
    vars('x','y')
    f = x^3-y+2*x^2*y
    g = x^2+x*y+y^2
    fx = DD(f,x):autosimplify()
    fy = DD(f,y):autosimplify()
    gx = DD(g,x):autosimplify()
    gy = DD(g,y):autosimplify()
    lagrange = fx*gy-fy*gx
\end{CAS}

\luaexec{
    tex.print("\\def\\g{",tostring(g),"}")
    tex.print("\\def\\l{",tostring(lagrange),"}")
}


\yoink{lagrange}{lag}

\[ \begin{aligned}
    f   &= \print{f} \\ 
    f_x &= \print{fx} \\
    f_y &= \print{fy}
\end{aligned} \] 

\[ \print{lagrange:expand()} \] 
\[ \print{g} \] 


\begin{asypicture}{width=7cm}
    import settings;
    settings.tex = "xelatex";
    import graph;
    import contour;
    size(7cm,6cm,IgnoreAspect);

    real G(pair uv){
        real x=uv.x; real y=uv.y;
        return @g;
    }
    real L(pair uv){
        real x=uv.x; real y=uv.y;
        return @lag;
    }

    real[] gvals={1};
    real[] lvals={0};
    
    guide[][] con = contour(G,(-2,-2),(2,2),gvals);
    guide[][] lag = contour(L,(-2,-2),(2,2),lvals);

    limits((-2.5,-2.5),(2.5,2.5));

    draw(con,blue+1);
    draw(lag,red+1);

    xaxis(Label("$x$",align=S),LeftRight,LeftTicks(Label(align=right)));
    yaxis(Label("$y$",align=W),BottomTop,RightTicks(Label(align=left)));
\end{asypicture}

\begin{CAS}
    a = 2*x*y
    b = (-1)*x*y
    c = a+b
\end{CAS}

$\print*{c} = \print{c}$


\parseforest{c}
\begin{forest}
    @\forestresult
\end{forest}


\luaexec{
    term1 = c.expressions[1]:autosimplify()
    term2 = c.expressions[2]:autosimplify()
}

\parseforest{term1}
\begin{forest}
    @\forestresult
\end{forest} \qquad \parseforest{term2} \qquad \begin{forest}
    @\forestresult
\end{forest} 

\luaexec{
    a = x
    if a.operation ~= BinaryOperation.MUL then
        a = Integer.one()*a
    end
}
\parseforest{a}
\begin{forest}
    @\forestresult
\end{forest}

$\print{a}$

\begin{CAS}
    a = x*y
    aa = BinaryOperation(BinaryOperation.MUL,{Integer.one(),a})
    b = 2*x*y
    c = 3*x*y
\end{CAS}
$\print{a+b+c}$

\parseforest{aa+b:autosimplify()}
\begin{forest}
    @\forestresult
\end{forest}

\luaexec{
    if ArrayEqual(aa.expressions,b:autosimplify().expressions,2) then 
        tex.print("Equal")
    else
        tex.print("Unequal")
    end
}

\begin{CAS}
    vars('x','y')
    c = sqrt(4*x^2)^2
\end{CAS}
\[ \print*{c} = \print{c} \] 

\begin{CAS}
a = (2/3)/x
\end{CAS}
\[ \print*{a} = \print{a} \] 

\begin{CAS}
    f = x^3-2
    f = f:autosimplify()
    f = f:topolynomial()
    r = f:roots()
\end{CAS}
\[ \print{r[1]} \] 

\begin{CAS}
    a = sqrt(2)
    b = 1/sqrt(2)
    c = a*b
\end{CAS}
\[ \print*{c} = \print{c} \] 

\begin{CAS}
    a = 1/tan(44*PI/3)
\end{CAS}
\[ \print*{a} = \print{a} \] 
\end{document}

