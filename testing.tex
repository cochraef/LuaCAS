\documentclass{article}
\usepackage{luacas}
\usepackage{amsmath}

\usepackage{xcolor}
\usepackage{parskip}

\begin{document}


\section{Core}
Testing \verb|core| commands. 

\subsection{Expressions}

\subsubsection{substitute}

$\luaexec{
    local sym1 = parse("x^2+x")
    local subs ={[SymbolExpression("x")]=SymbolExpression("y")}
    local sym2 = sym1:substitute(subs)
    tex.print(sym2:tolatex())
}$

    
\subsubsection{isatomic} 
Symbol expressions are by default \verb|AtomicExpressions|. For example: 
\begin{quote} 
If we store \verb|asym = SymbolExpression("x^2+x")| then the sentence ``the expression \verb|asym| is an atomic expression'' is 
\luaexec{
    local asym = SymbolExpression("x^2+x")
    if asym:isatomic() then
        tex.print("True")
    else
        tex.print("False")
    end
}.
\end{quote}
But expressions that are constructed using \verb|BinaryOperations| are non-atomic. For example:
\begin{quote}
    If we store \verb|x = SymbolExpression("x")| and \verb|csym = x*x+x| then the sentence ``the expression \verb|csym| is an atomic expression'' is
    \luaexec{
    local x = SymbolExpression("x")
    local csym = x*x+x
    if csym:isatomic() then
        tex.print("True")
    else
        tex.print("False")
    end
    }.
\end{quote}
We also have:
\begin{quote}
    If we store \verb|csym = parse("x^2+x")| then the sentence ``the expression \verb|csym| is an atomic expression'' is
    \luaexec{
    local csym = parse("x^2+x")
    if csym:isatomic() then
        tex.print("True")
    else
        tex.print("False")
    end
    }.
\end{quote}



\subsection{symbolexpression.lua} 
Testing commands in \verb|symbolexpression.lua|. 

\subsubsection{evaluate}

\subsubsection{freeof} 
The statement ``$x^2+x+3$ is free of $x$'' is: %
\luaexec{
    local x = SymbolExpression("x")
    local sym = x*x+x+Integer(3)
    local y = SymbolExpression("x")
    if sym:freeof(y) then
        tex.print("True")
    else 
        tex.print("False")
    end
}

The statement ``$x^2+x+3$ is free of $y$'' is: %
\luaexec{
    local x = SymbolExpression("x")
    local sym = x*x+x+Integer(3)
    local y = SymbolExpression("y")
    if sym:freeof(y) then
        tex.print("True")
    else 
        tex.print("False")
    end
}

\subsubsection{expand}

Using the distributive property, we have:
$\luaexec{
    x = SymbolExpression("x")
    y = SymbolExpression("y")
    sym = (x-Integer(1))*(y+Integer(2))
    sym2 = parse("(x-1)*(y+2)")
    esym= sym2:expand()
    tex.print(sym2:tolatex(),"=",esym:tolatex())
}$. {\color{red} It would be good for $2x+-y$ to be written as $2x-y$.}


\newpage

\section{Bugs \& Feature Requests}
{\color{gray} 
\begin{enumerate}
\item There's an issue with parentheses in the \verb|tolatex| command:
\[ \CAS{
    disp(1/((h-1)^2))
} \] 
\item We also might want to add the ability for \verb|\CAS| to identify \verb|local| variables. Let's add \verb|exp| as a known function:
\[ \CAS{ disp(int(z*e^(z^2),z)) } \qquad \text{vs} \qquad \CAS{ disp(int(z*exp(z^2),z)) } \] 
\item It might be nice to be able to use Lua logical operators in \verb|\CAS|;  and also to declare local variables to limit scope.
\item Glorified \verb|simplify| command might be good (for getting rational expressions in a common denominator, or combining logs and such).
\end{enumerate}
}
\[ \CAS{
    local g = 2;
    disp(g);
    } \] 
\CAS{disp(g)}
\luaexec{
CASparse("g=3; h=x^2+x")
disp(g)
if h==SymbolExpression("x^2+x") then tex.print("Symbols!") end
}

\end{document}

