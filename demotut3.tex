\documentclass{article}
\usepackage{multicol,fullpage,luacas,parskip,asypictureB}

\usepackage{pgfplots,forest}
\usetikzlibrary{calc}

\begin{document}

\begin{CAS}
    vars('x','h')
    f = 1/(x^2+1)
    subs = {[x]=x+h}
    q = (substitute(subs,f)-f)/h
    q = expand(q)
\end{CAS}

\parseforest{q}
\bracketset{action character = @}
\begin{forest}
    for tree = {
        font=\ttfamily,
        rectangle,
        rounded corners=1pt
    },
    where level=0{%
        fill=orange!25
    }{},
    @\forestresult
\end{forest}

\directlua{
    if q.operation == BinaryOperation.ADD then 
        tex.print("I'm an \\texttt{ADD}")
    end
}

\luaexec{
    den = {}
    num = {}
    auxden = {}
    if q.operation == BinaryOperation.ADD then 
        for _, expr in ipairs(q.expressions) do
            local numpart = Integer.one()
            local denpart = Integer.one()
            if expr.operation == BinaryOperation.MUL then 
                for _,subexpr in ipairs(expr.expressions) do 
                    if subexpr.operation == BinaryOperation.POW and subexpr.expressions[2] == -Integer.one() then
                        denpart = denpart*subexpr.expressions[1]    
                        for _,term in ipairs(den) do 
                            if subexpr.expressions[1] == term then 
                                goto continue
                            end
                        end
                        table.insert(den,subexpr.expressions[1])
                        ::continue::
                    else 
                        numpart = numpart*subexpr 
                    end
                end
            end
            if expr.operation == BinaryOperation.POW and expr.expressions[2] == -Integer.one() then 
                denpart = denpart*expr.expressions[1]
                for _,term in ipairs(den) do 
                    if expr.expressions[1] == term then 
                        goto continue
                    end
                end
                table.insert(den,subexpr.expressions)
                ::continue::
            end
            table.insert(num,numpart)
            table.insert(auxden,denpart)
        end
    end
    denominator = Integer.one()
    numerator   = Integer.zero()
    for _,expr in ipairs(den) do 
        denominator = denominator*expr
    end
    denominator = denominator:simplify()
    for index,expr in ipairs(num) do
        local numer = denominator/auxden[index]
        numer = numer:simplify() 
        numerator = numerator + expr*numer
    end
    numerator = numerator:simplify():expand():factor()
    common = numerator/denominator
}
\[ \print{denominator} \] 
\[ \print{numerator} \] 
\[ \print{common} \] 



\end{document}