\documentclass{article}

\usepackage{luacas}

\usepackage[margin=1in]{geometry}
\usepackage[shortlabels]{enumitem}

\usepackage{tikz}
\usepackage{minted}
%\usemintedstyle{pastie}
\usepackage[hidelinks]{hyperref}
\usepackage{parskip}
\usepackage{multicol}
\usepackage[most]{tcolorbox}
\usepackage{microtype}


\newtcolorbox{codebox}[1][sidebyside]{
    enhanced,skin=bicolor,
    #1,
    arc=1pt,
    colframe=brown,
    colback=brown!15,colbacklower=white,
    boxrule=1pt,
    notitle
}

\begin{document}
\title{A portable Computer Algebra System capable of symbolic computation for use in Lua\LaTeX{}: \\  The {\ttfamily luacas} package }
\author{Evan Cochrane \\ {\itshape Timothy All}}
\date{} 

\maketitle 

\tableofcontents

\section{Introduction}

The package {\ttfamily luacas} allows for symbolic computation within \LaTeX{}. For example:
\begin{CAS}
    vars('x','y')
    f = 3*x*y - x^2*y 
    fx = diff(f,x)
\end{CAS}
\inputminted[
firstline=7,
lastline=11,
breaklines]
{latex}
{demointro.tex}
The above code will compute the partial derivative $f_x$ of the function $f$ defined by 
\[ f(x,y)=3xy-x^2y.\]
There are various methods for fetching and/or printing results from the CAS within your \LaTeX{} document:

\begin{codebox}
\inputminted[
firstline=14,
lastline=14,
breaklines]
{latex}
{demointro.tex}
\tcblower
\[ \print*{fx} = \print{fx} \] 
\end{codebox}

\subsection{About}

The core CAS program is written purely in Lua and integrated into \LaTeX{} via Lua\LaTeX{}. Currently, most existing computer algebra systems such as Maple and Mathematica allow converting their stored expressions to \LaTeX{} code, but this still requires exporting code from \LaTeX{} to another program and importing it back, which can be tedious.

The target audience for this package are mathematics students, instructors, and professionals who would like some ability to perform basic symbolic computations within \LaTeX{} without the need for laborious and technical setup. But truly, this package was born out of a desire from the authors to learn more about symbolic computation. What you're looking at here is the proverbial ``carrot at the end of the stick'' to keep our learning moving forward.
                
Using a scripting language (like Lua) for the core CAS reduces performance dramatically, but the following considerations make it a good option for our intentions: 
                
\begin{itemize}
    \item Compiled languages that can communicate with \LaTeX{} in some way (such as C through Lua) require compiling the code on each machine before running, reducing portability.
    \item Our target usage would generally not involve computations that take longer than a second, such as factoring large primes or polynomials.
    \item Lua is a fast scripting language, especially when compared to Python, and is designed to be compact and portable.
    \item If C code could be used, we could tie into one of many open-source C symbolic calculators, but the point of this project is to learn the mathematics of symbolic computation. The barebones but friendly nature of Lua make it a fairly ideal language from a pedagogical point of view.
\end{itemize}

\subsection{Features}
                
Currently, {\ttfamily luacas} includes the following functionality:
                
\begin{itemize}
    \item Arbitrary-precision integer and rational arithmetic
    \item Number-theoretic algorithms for factoring integers and determining primality
    \item Constructors for arbitrary polynomial rings and integer mod rings, and arithmetic algorithms for both
    \item Factoring univariate polynomials over the rationals and over finite fields
    \item Polynomial decomposition and some multivariate functionality, such as pseudodivision
    \item Basic symbolic root finding
    \item Symbolic expression manipulations such as expansion, substitution, and simplification
    \item Symbolic differentiation and integration
\end{itemize}
                
The CAS is written using object-oriented Lua, so it is modular and would be easy to extend its functionality (which we hope to do in the future).

\subsection{Acknowledgements}

We'd like to thank the faculty of the Department of Mathematics at Rose-Hulman Institute of Technology for offering constructive feedback as we worked on this project. In particular, a special thanks goes to Dr. Joseph Eichholz for his invaluable input and helpful suggestions. 

\section{Installation}

\subsection{Requirements}

The \texttt{luacas} package (naturally) requires you to compile with Lua\LaTeX{}. Beyond that, the following packages are needed:
\begin{multicols}{2}
{\ttfamily 
\begin{itemize}
    \item xparse
    \item pgfkeys
    \item verbatim
    \item amsmath
    \item luacode
    \item iftex
    \item tikz/forest
    \item xcolor
\end{itemize}}
\end{multicols}
The packages {\ttfamily tikz, forest, xcolor} aren't strictly required, but they are needed for drawing expression trees.

\subsection{Installing {\ttfamily luacas}}
The package manager for your local TeX distribution ought to install the package fine on its own. But for those who like to take matters into their own hands: unpack \texttt{luacas.zip} in the current working directory (or in a directory visible to TeX, like your local texmf directory), and in the preamble of your document, put:
\inputminted[firstline=3,
    lastline=3,
    breaklines]
    {latex}
    {doc.tex}
That's it, you're ready to go.

\section{Tutorials}

Taking a cue from the phenomenal TikZ documentation, we introduce basic usage of the \texttt{luacas} package through a few informal tutorials.  

\subsection{Tutorial 1: Limit Definition of the Derivative}

Alice is teaching calculus, and she wants to give her students many examples of the dreaded \emph{limit definition of the derivative}. On the other hand, she'd like to avoid working out many examples by-hand. She decides to give \texttt{luacas} a try.

Alice can access the \texttt{luacas} program using a custom environment: \mintinline{latex}{\begin{CAS}..\end{CAS}}. The first thing Alice must do is declare variables that will be used going forward:
\inputminted[
    firstline=6,
    lastline=8,
    breaklines]
    {latex}
    {demotut1.tex}
Alice decides that $f$, the function to be differentiated, should be $x^2$. So Alice makes this assignment with:
\inputminted[
    firstline=10,
    lastline=13,
    breaklines]
    {latex}
    {demotut1.tex}
Now, Alice wants to use the variable $q$ to store the appropriate \emph{difference quotient} of $f$. Alice could hardcode this into $q$, but that seems to defeat the oft sought after goal of reusable code. So Alice decides to use the \texttt{substitute} command of \texttt{luacas}:
\inputminted[
    firstline=15,
    lastline=20,
    breaklines]
    {latex}
    {demotut1.tex}
Alice is curious to know if $q$ is what she thinks it is. So Alice decides to have \LaTeX{} print out the contents of $q$ within her document. For this, she uses the \mintinline{latex}{\print} command. 
\begin{CAS}
    vars('x','h')
    f = x^2
    subs = {[x]=x+h}
    q = (substitute(subs,f)- f)/h
\end{CAS}
\begin{codebox}
    {\small\inputminted[
    firstline=22,
    lastline=22,
    breaklines]
    {latex}
    {demotut1.tex}
    }
    \tcblower
    \[ \print{q} \] 
\end{codebox}
So far so good! Alice wants to expand the numerator of $q$; she finds the aptly named \texttt{expand} method helpful in this regard. Alice redefines \mintinline{lua}{q} to be \mintinline{lua}{q=expand(q)}, and prints the result to see if things worked as expected:
\begin{codebox}
    {\small\inputminted[
    firstline=24,
    lastline=31,
    breaklines]
    {latex}
    {demotut1.tex}
    }
    \tcblower
    \begin{CAS}
        q = expand(q)
    \end{CAS}
    \[ \print*{q} \] 
\end{codebox}
Alice is pleasantly surprised that the result of the expansion has been \emph{autosimplified}, i.e., the factors of $x^2$ and $-x^2$ cancelled each other out, and the resulting extra factor of $h$ has been cancelled out of the numerator and denominator.

Finally, Alice wants to take the limit as $h\to 0$. Now that our difference quotient has been expanded and simplified (automatically), this amounts to another substitution:
\begin{codebox}
    {\small\inputminted[
    firstline=33,
    lastline=42,
    breaklines]
    {latex}
    {demotut1.tex}
    }
    \tcblower
    \begin{CAS}
        subs = {[h]=0}
        q = q:substitute(subs)
    \end{CAS}
    \[ \print{q} \] 
\end{codebox}
Alice is slightly disappointed that $0+2x$ is returned and not $2x$. Alice takes a guess that there's a \mintinline{lua}{simplify} command. This does the trick: adding the line \mintinline{lua}{q = simplify(q)} before leaving the \texttt{CAS} environment returns the expected $2x$:
\begin{codebox}
    {\small\inputminted[
    firstline=44,
    lastline=54,
    breaklines]
    {latex}
    {demotut1.tex}
    }
    \tcblower
    \begin{CAS}
        q=simplify(q)
    \end{CAS}
    \[ \print{q} \] 
\end{codebox}

Alternatively, Alice could have used the \mintinline{latex}{\print*} command instead of \mintinline{latex}{\print} -- the essential difference is that \mintinline{latex}{\print*}, unlike \mintinline{latex}{\print}, automatically simplifies the content of the argument. 

Alice is pretty happy with how everything is working, but she wants to be able to typeset the individual steps of this process. Alice is therefore thrilled to learn that the \mintinline{latex}{\begin{CAS}..\end{CAS}} environment is very robust -- it can:
\begin{itemize}
    \item Be entered into and exited out of essentially anywhere within her \LaTeX{} document, for example, within \mintinline{latex}{\begin{aligned}..\end{aligned}}; and 
    \item CAS macros persist -- if Alice assigns \mintinline{lua}{f = x^2} within \mintinline{latex}{\begin{CAS}..\end{CAS}}, then the CAS remembers that \mintinline{lua}{f = x^2} the next time Alice enters the CAS environment. 
\end{itemize}
Here's the final version of Alice's code: 
\begin{codebox}[sidebyside align=top]
    {\small\inputminted[
    firstline=56,
    lastline=82,
    breaklines]
    {latex}
    {demotut1.tex}
    }
    \tcblower
    \begin{CAS}
        vars('x','h')
        f = x^2
    \end{CAS}
    Let $f(x) = \print{f}$. We wish to compute the derivative of $f(x)$ at $x$ using the limit definition of the derivative. Toward that end, we start with the appopriate difference quotient:
    \begin{CAS}
        subs = {[x]=x+h}
        q = (substitute(subs,f) - f)/h
    \end{CAS}
    \[ \begin{aligned}
        \print{q} &= 
        \begin{CAS} 
            q = expand(q) 
        \end{CAS}
        \print{q}& &\text{expand/simplify} \\
        \begin{CAS}
            subs = {[h]=0}
            q = substitute(subs,q)
        \end{CAS}
        &\xrightarrow{h\to 0} \print{q}& &\text{take limit}\\ 
        &= 
        \begin{CAS}
            q = simplify(q)
        \end{CAS}
        \print{q} & &\text{simplify.}
    \end{aligned} \] 
    So $\print{diff(f,x)} = \print*{diff(f,x)}$.     
\end{codebox}
Alice can produce another example merely by changing the definition of $f$ on the third line to another polynomial:
\begin{codebox}[sidebyside align=top]
    {\small\inputminted[
    firstline=84,
    lastline=88,
    breaklines]
    {latex}
    {demotut1.tex}
    }
    \tcblower
    \begin{CAS}
        vars('x','h')
        f = 2*x^3-x
    \end{CAS}
    Let $f(x) = \print{f}$. We wish to compute the derivative of $f(x)$ at $x$ using the limit definition of the derivative. Toward that end, we start with the appopriate difference quotient:
    \begin{CAS}
        subs = {[x] = x+h}
        q = (f:substitute(subs) - f)/h
    \end{CAS}
    \[ \begin{aligned}
        \print{q} &= 
        \begin{CAS} 
            q = expand(q)
        \end{CAS}
        \print{q}& &\text{expand/simplify} \\
        \begin{CAS}
            subs = {[h]=0}
            q = q:substitute(subs)
        \end{CAS}
        &\xrightarrow{h\to 0} \print{q}& &\text{take limit} \\ 
        &= 
        \begin{CAS}
            q = simplify(q)
        \end{CAS}
        \print{q}& &\text{simplify.}
    \end{aligned} \] 
    So $\print{diff(f,x)} = \print*{diff(f,x)}$.     
\end{codebox}

\subsection{Tutorial 2: Finding maxima/minima}

Bob is teaching calculus too, and he wants to give his students many examples of \emph{finding the local max/min of a given function}. But, like Alice, Bob doesn't want to work out a bunch of examples by-hand. Bob decides to try his hand with \texttt{luacas} after having been taught the basics by Alice. 

Bob decides to stick with polynomials for these examples; if anything because those functions are in the wheel-house of \texttt{luacas}. In particular, Bob decides that the \emph{derivative} of the function he wants to use should be a composition of quadratics. This ought to ensure that the roots of that derivative are expressible in a nice way. 

Accordingly, Bob declares variables and chooses two quadratic polynomials to compose, say $f$ and $g$, and sets $dh = g \circ f$:

\inputminted[
firstline=9,
lastline=15,
breaklines]
{latex}
{demotut2.tex}

Bob wants to compute $h$, the integral of $dh$. Bob could certainly compute this quantity by-hand, but why hardcode that information into the document when \texttt{luacas} can do this for you? So Bob uses the \texttt{int} command:

\inputminted[
firstline=17,
lastline=19,
breaklines]
{latex}
{demotut2.tex}

Bob is curious to know the value of $h$. So he uses the \mintinline{latex}{\print{h}} to produce:
\begin{codebox}
    {\small\inputminted[
        firstline=21,
        lastline=21,
        breaklines]
        {latex}
        {demotut2.tex}}
    \tcblower
    \begin{CAS}
        vars('x')
        f = x^2+2*x-2
        g = x^2-1
        subs = {[x] = f}
        dh = substitute(subs,g)
        h  = int(dh,x)
    \end{CAS}
    \[\print{h} \] 
\end{codebox}
This isn't exactly what Bob had in mind. It occurs to Bob that he may need to simplify the expression $h$, so he tries:

\begin{codebox}
    {\small\inputminted[
        firstline=23,
        lastline=26,
        breaklines]
        {latex}
        {demotut2.tex}
    }
    \tcblower
    \begin{CAS}
        h = simplify(h)
    \end{CAS}
    \[\print{h} \] 
\end{codebox}

That's more like it! Now, Bob wants to find the roots to $dh$. Bob uses the \texttt{roots} command to do this:

\inputminted[
    firstline=28,
    lastline=30,
    breaklines]
    {latex}
    {demotut2.tex}

But then Bob wonders to himself, ``How do I actually retrieve the roots of $dh$ from \texttt{luacas}?'' The assignment \mintinline{lua}{r = roots(dh)} stores the roots of the polynomial $dh$ in a table named \texttt{r}. With indices beginning at $1$, Bob can retrieve the values of these roots via:

\begin{codebox}
    \inputminted[
    firstline=32,
    lastline=32,
    breaklines]
    {latex}
    {demotut2.tex}
    \tcblower 
    \begin{CAS}
        r = roots(dh)
    \end{CAS}
    \[ \print{r[1]}, \quad \print{r[2]}, \quad \print{r[3]}, \quad \print{r[4]} \] 
\end{codebox}

Splendid! Bob would now like to evaluate the function $h$ at these roots (for these are the local max/min values of $h$). Here's Bob's first thought:

\begin{codebox}
    \inputminted[
    firstline=34,
    lastline=37,
    breaklines]
    {latex}
    {demotut2.tex}
    \tcblower 
    \begin{CAS}
        v1 = simplify(substitute({[x]=r[1]},h))
    \end{CAS}
    \[ \print{v1} \] 
\end{codebox}

What's going on? Bob is (understandably) confused. But here's where Bob learns a valuable lesson\dots 

\subsubsection*{Lua numbers vs \texttt{luacas} Integers}



\subsection{Advanced}



\appendix

\section{The \LaTeX{} code}

As noted above, this package is really a Lua program; the package {\ttfamily luacas.sty} is merely a shell to make accessing that Lua program easy and manageable from within \LaTeX{}. 
\inputminted[
    firstline=12,
    lastline=14,
    breaklines,
    linenos,
    numbersep=5pt]
    {latex}
    {luacas.sty}

We check to make sure the user is compiling with Lua\LaTeX{}; if not, an error message is printed and compilation is aborted. 

\inputminted[
    firstline=16,
    lastline=24,
    breaklines,
    linenos,
    numbersep=5pt]
    {latex}
    {luacas.sty}

The following pacakages are required for various macros:

\inputminted[
    firstline=27,
    lastline=32,
    breaklines,
    linenos,
    numbersep=5pt]
    {latex}
    {luacas.sty}

The files \verb|helper.lua| and \verb|parser.lua| help bridge the gap between the Lua program and \LaTeX{}. 

\inputminted[
    firstline=35,
    lastline=37,
    breaklines,
    linenos,
    numbersep=5pt]
    {latex}
    {luacas.sty}

\end{document}