\documentclass{article}

\usepackage{luacas}

\usepackage[margin=1in]{geometry}
\usepackage[shortlabels]{enumitem}

\usepackage{tikz}
\usepackage{minted}
%\usemintedstyle{pastie}
\usepackage[hidelinks]{hyperref}
\usepackage{parskip}
\usepackage{multicol}
\usepackage[most]{tcolorbox}
\usepackage{microtype}


\newtcolorbox{codebox}[1][sidebyside]{
    enhanced,skin=bicolor,
    #1,
    arc=1pt,
    colframe=brown,
    colback=brown!15,colbacklower=white,
    boxrule=1pt,
    notitle
}

\begin{document}
\title{A portable Computer Algebra System capable of symbolic computation for use in Lua\LaTeX{}: \\  The {\ttfamily luacas} package }
\author{Evan Cochrane \\ {\itshape Timothy All, Joe Eichholz}}
\date{} 

\maketitle 

\tableofcontents

\section{Introduction}

The package {\ttfamily luacas} allows for symbolic computation within \LaTeX{}. For example:
\begin{CAS}
    vars('x','y')
    f = 3*x*y - x^2*y 
    fx = diff(f,x)
\end{CAS}
\inputminted[
firstline=7,
lastline=11,
breaklines]
{latex}
{docdemo.tex}
The above code will compute the partial derivative $f_x$ of the function $f$ defined by 
\[ f(x,y)=3xy-x^2y.\]
There are various methods for fetching and/or printing results from the CAS within your \LaTeX{} document:

\begin{codebox}
\inputminted[
firstline=14,
lastline=14,
breaklines]
{latex}
{docdemo.tex}
\tcblower
\[ \print*{fx} = \print{fx} \] 
\end{codebox}

\subsection{About}

The core CAS program is written purely in Lua and integrated into \LaTeX{} via Lua\LaTeX{}. Currently, most existing computer algebra systems such as Maple and Mathematica allow converting their stored expressions to \LaTeX{} code, but this still requires exporting code from \LaTeX{} to another program and importing it back, which can be tedious.

The target audience for this package are mathematics students, instructors, and professionals who would like some ability to perform basic symbolic computations within \LaTeX{} without the need for laborious and technical setup. But truly, this package was born out of a desire from the authors to learn more about symbolic computation. What you're looking at here is the proverbial ``carrot at the end of the stick'' to keep our learning moving forward.
                
Using a scripting language (like Lua) for the core CAS reduces performance dramatically, but the following considerations make it a good option for our intentions: 
                
\begin{itemize}
    \item Compiled languages that can communicate with \LaTeX{} in some way (such as C through Lua) require compiling the code on each machine before running, reducing portability.
    \item Our target usage would generally not involve computations that take longer than a second, such as factoring large primes or polynomials.
    \item Lua is a fast scripting language, especially when compared to Python, and is designed to be compact and portable.
    \item If C code could be used, we could tie into one of many open-source C symbolic calculators, but the point of this project is to learn the mathematics of symbolic computation. The barebones but friendly nature of Lua make it a fairly ideal language from a pedagogical point of view.
\end{itemize}

\subsection{Features}
                
Currently, {\ttfamily luacas} includes the following functionality:
                
\begin{itemize}
    \item Arbitrary-precision integer and rational arithmetic
    \item Number-theoretic algorithms for factoring integers and determining primality
    \item Constructors for arbitrary polynomial rings and integer mod rings, and arithmetic algorithms for both
    \item Factoring univariate polynomials over the rationals and over finite fields
    \item Polynomial decomposition and some multivariate functionality, such as pseudodivision
    \item Basic symbolic root finding
    \item Symbolic expression manipulations such as expansion, substitution, and simplification
    \item Symbolic differentiation and integration
\end{itemize}
                
The CAS is written using object-oriented Lua, so it is modular and would be easy to extend its functionality (which we hope to do in the future).

\section{Installation}

\subsection{Requirements}

The \texttt{luacas} package (naturally) requires you to compile with Lua\LaTeX{}. Beyond that, the following packages are needed:
\begin{multicols}{2}
{\ttfamily 
\begin{itemize}
    \item xparse
    \item pgfkeys
    \item verbatim
    \item amsmath
    \item luacode
    \item iftex
    \item tikz/forest
    \item xcolor
\end{itemize}}
\end{multicols}
The packages {\ttfamily tikz, forest, xcolor} aren't strictly required, but they are needed for drawing expression trees.

\subsection{Installing {\ttfamily luacas}}
The package manager for your local TeX distribution ought to install the package fine on its own. But for those who like to take matters into their own hands: unpack \texttt{luacas.zip} in the current working directory (or in a directory visible to TeX, like your local texmf directory), and in the preamble of your document, put:
\inputminted[firstline=3,
    lastline=3,
    breaklines]
    {latex}
    {doc.tex}
That's it, you're ready to go.

\section{Tutorials}

Taking a cue from the phenomenal TikZ documentation, we introduce basic usage of the \texttt{luacas} package through a few informal tutorials.  

\subsection{Tutorial 1: Limit Definition of the Derivative}

Alice is teaching calculus, and she wants to give her students many examples of the dreaded \emph{limit definition of the derivative}. On the other hand, she'd like to avoid working out many examples by-hand. She decides to give \texttt{luacas} a try.

Alice can access the \texttt{luacas} program using a custom environment: \mintinline{latex}{\begin{CAS}..\end{CAS}}. The first thing Alice must do is declare variables that will be used going forward:
\inputminted[
    firstline=55,
    lastline=57,
    breaklines]
    {latex}
    {demo.tex}
Alice decides that $f$, the function to be differentiated, should be $x^2$. So Alice makes this assignnment with:
\inputminted[
    firstline=59,
    lastline=62,
    breaklines]
    {latex}
    {demo.tex}
Now, Alice wants to use the variable $q$ to store the appropriate \emph{difference quotient} of $f$. Alice could hardcode this into $q$, but that seems to defeat the oft sought after goal of reusable code. So Alice decides to use the \texttt{substitute} command of \texttt{luacas}:
\inputminted[
    firstline=64,
    lastline=69,
    breaklines]
    {latex}
    {demo.tex}
Alice is curious to know if $q$ is what she thinks it is. So Alice decides to have \LaTeX{} print out the contents of $q$. For this, she uses the \mintinline{latex}{\print*} command. 
\begin{CAS}
    vars('x','h')
    f = x^2
    subs = {[x]=x+h}
    q = (substitute(subs,f)- f)/h
\end{CAS}
\begin{codebox}
    \inputminted[
    firstline=71,
    lastline=71,
    breaklines]
    {latex}
    {demo.tex}
    \tcblower
    \[ \print*{q} \] 
\end{codebox}
So far so good! Alice wants to expand the numerator of $q$; she finds the aptly named \texttt{expand} method helpful in this regard. Alice redefines \mintinline{lua}{q} to be \mintinline{lua}{q=expand(q)}, and prints the result to see if things worked as expected:
\begin{codebox}
    \inputminted[
    firstline=73,
    lastline=80,
    breaklines]
    {latex}
    {demo.tex}
    \tcblower
    \begin{CAS}
        q = expand(q)
    \end{CAS}
    \[ \print*{q} \] 
\end{codebox}
Alice is pleasantly surprised that the result of the expansion has been autosimplified, i.e., the $x^2$ and $-x^2$ came to $0$ which was omitted from the numerator of the expression altogether. Now, Alice wants to distribute the $h$ in the denominator; this requires another expansion: 
\begin{codebox}
    \inputminted[
    firstline=82,
    lastline=90,
    breaklines]
    {latex}
    {demo.tex}
    \tcblower
    \begin{CAS}
        q = expand(q)
    \end{CAS}
    \[ \print*{q} \] 
\end{codebox}
Finally, Alice wants to take the limit as $h\to 0$. Now that our difference quotient has been expanded (twice) and simplified (automatically), this amounts to another substitution:
\begin{codebox}
    \inputminted[
    firstline=92,
    lastline=102,
    breaklines]
    {latex}
    {demo.tex}
    \tcblower
    \begin{CAS}
        subs = {[h]=0}
        q = q:substitute(subs)
    \end{CAS}
    \[ \print*{q} \] 
\end{codebox}
Alice is slightly disappointed that $0+2x$ is returned and not $2x$. Alice takes a guess that there's a \mintinline{lua}{simplify} command. This does the trick: adding the line \mintinline{lua}{q = simplify(q)} before leaving the \texttt{CAS} environment returns the expected $2x$.

Alternatively, Alice could have used the \mintinline{latex}{\print} command instead of \mintinline{latex}{\print*} -- the essential difference is that \mintinline{latex}{\print}, unlike \mintinline{latex}{\print*}, automatically applies the \mintinline{lua}{:simplify()} method the content of the argument. 

Alice is pretty happy with how everything is working, but she wants to be able to typeset the individual steps of this process. Alice is therefore thrilled to learn that the \mintinline{latex}{\begin{CAS}..\end{CAS}} environment is very robust -- it can:
\begin{itemize}
    \item be entered into and exited out of essentially anywhere within her \LaTeX{} document, for example, within \mintinline{latex}{\begin{aligned}..\end{aligned}}; and 
    \item CAS macros persist -- if Alice assigns \mintinline{lua}{f = x^2} within \mintinline{latex}{\begin{CAS}..\end{CAS}}, then the CAS remembers that \mintinline{lua}{f = x^2} the next time Alice enters the CAS environment. 
\end{itemize}
Here's the final version of Alice's code: 
\begin{codebox}[sidebyside align=top]
    \inputminted[
    firstline=117,
    lastline=143,
    breaklines]
    {latex}
    {demo.tex}
    \tcblower
    \begin{CAS}
        vars('x','h')
        f = x^2
    \end{CAS}
    Let $f(x) = \print{f}$. We wish to compute the derivative of $f(x)$ at $x$ using the limit definition of the derivative. Toward that end, we start with the appopriate difference quotient:
    \begin{CAS}
        subs = {[x] = x+h}
        q = (f:substitute(subs) - f)/h
    \end{CAS}
    \[ \begin{aligned}
        \print*{q} &= 
        \begin{CAS} 
            q = q:expand() 
        \end{CAS}
        \print*{q}& &\text{expand/simplify numerator} \\
        &= 
        \begin{CAS}
            q = q:expand() 
        \end{CAS}
        \print*{q}& &\text{distribute denominator} \\ 
        \begin{CAS}
            subs = {[h]=0}
            q = q:substitute(subs)
        \end{CAS}
        &\xrightarrow{h\to 0} \print{q}& &\text{take limit.}
    \end{aligned} \] 
    So $\print*{diff(f,x)} = \print{diff(f,x)}$. 
\end{codebox}
Alice can produce another example merely by changing the definition of $f$ on the third line to another polynomial:
\begin{codebox}[sidebyside align=top]
    \inputminted[
    firstline=145,
    lastline=150,
    breaklines]
    {latex}
    {demo.tex}
    \tcblower
    \begin{CAS}
        vars('x','h')
        f = 2*x^3-x
    \end{CAS}
    Let $f(x) = \print*{f}$. We wish to compute the derivative of $f(x)$ at $x$ using the limit definition of the derivative. Toward that end, we start with the appopriate difference quotient:
    \begin{CAS}
        subs = {[x] = x+h}
        q = (f:substitute(subs) - f)/h
    \end{CAS}
    \[ \begin{aligned}
        \print*{q} &= 
        \begin{CAS} 
            q = q:expand() 
        \end{CAS}
        \print*{q}& &\text{expand/simplify numerator} \\
        &= 
        \begin{CAS}
            q = q:expand() 
        \end{CAS}
        \print*{q}& &\text{distribute denominator} \\ 
        \begin{CAS}
            subs = {[h]=0}
            q = q:substitute(subs)
        \end{CAS}
        &\xrightarrow{h\to 0} \print{q}& &\text{take limit.}
    \end{aligned} \] 
    So $\print*{diff(f,x)} = \print{diff(f,x)}$.  
\end{codebox}



\subsection{Intermediate}

\subsection{Advanced}



\appendix

\section{The \LaTeX{} code}

As noted above, this package is really a Lua program; the package {\ttfamily luacas.sty} is merely a shell to make accessing that Lua program easy and manageable from within \LaTeX{}. 
\inputminted[
    firstline=12,
    lastline=14,
    breaklines,
    linenos,
    numbersep=5pt]
    {latex}
    {luacas.sty}

We check to make sure the user is compiling with Lua\LaTeX{}; if not, an error message is printed and compilation is aborted. 

\inputminted[
    firstline=16,
    lastline=24,
    breaklines,
    linenos,
    numbersep=5pt]
    {latex}
    {luacas.sty}

The following pacakages are required for various macros:

\inputminted[
    firstline=27,
    lastline=32,
    breaklines,
    linenos,
    numbersep=5pt]
    {latex}
    {luacas.sty}

The files \verb|helper.lua| and \verb|parser.lua| help bridge the gap between the Lua program and \LaTeX{}. 

\inputminted[
    firstline=35,
    lastline=37,
    breaklines,
    linenos,
    numbersep=5pt]
    {latex}
    {luacas.sty}
    
\section{Lua Documentation}

This section assumes some familiarity with the programming language Lua and with object-oriented programming. For a comprehensive introduction to Lua, we refer the reader to \cite{pil}. This section is geared towards for those who are interested in the inner design, as opposed to the algorithms or mathematics, of the CAS. In particular, anyone who wishes to modify or expand the CAS for their own purposes should read this.

This computer algebra system is written in object-oriented Lua. Lua does not have built-in object-oriented features, but using metatables we can simulate much of the oo functionality of a language like Java. Every class in the CAS inherits from the {\ttfamily Expression} interface. Expressions are really trees, and may have any number of sub-expressions as children depending on the type of the expression.

Most operations that a user would want to apply to expressions, such as derivatives and integrals, are encapsulated into their own classes. These classes have the expressions that these operations are applied to as instance variables. The other kind of operations on expressions are \emph{core methods}, which are methods in the objected-oriented sense, and can be called on expressions using the {\ttfamily :} syntax as is typical in Lua. The rule of thumb is that if there is standard mathematical notation for an operation or function represented by the CAS, it gets its own class, and if not, it is a core method. This way, every Lua expression corresponds to something that can be cleanly displayed in \LaTeX{}.


\subsection{Core Methods}

All of the following core methods can be applied to any expression, and are defined in the {\ttfamily expression.lua} interface.

\begin{itemize}
    \item {\ttfamily evaluate()} applies an expression that represents an operation on expressions to its subexpressions. For instance, evaluating a {\ttfamily DerrivativeExpression} applies the derivative operator with respect to the {\ttfamily symbol} field to its {\ttfamily expression} field. Evaluating a {\ttfamily BinaryOperation} with its {\ttfamily operation} field set to {\ttfamily ADD} returns the sum of the numbers in the {\ttfamily expressions} field, if all of the expressions are numbers. If the expression does not represent an operation or is unable to be evaluated, calling {\ttfamily evaluate()} on an expression returns itself.
    
    \item {\ttfamily autosimplify()} performs fast simplification techniques on an expression. All input from the \LaTeX{} end and everything output to the \LaTeX{} end is autosimplified. Generally, one should call {\ttfamily autosimplify()} on expressions before applying other core methods to them. The {\ttfamily autosimplify()} method also calls the {\ttfamily evaluate()} method.
    
    \item {\ttfamily simplify()} performs full simplification of an expression. This may involve polynomial factorization, which is slow, so this is seperate from autosimplification and not called on the \LaTeX{} end unless the user specifically directs the CAS to simplify.
    
    \item {\ttfamily subexpressions()} returns a list of all subexpressions of an expression. This gives a unified interface to the fields of objects, which have different names across classes.
    
    \item {\ttfamily size()} returns the number of nodes of the tree that constitutes an expression, roughly or the total number of expression objects that make up the expression.
    
    \item {\ttfamily setsubexpressions(subexpressions)} creates a copy of an expression with the list of subexpressions as its new subexpressions. This can reduce code duplication in other methods.
    
    \item {\ttfamily freeof(symbol)} determines whether or not an expression contains a particular \\ {\ttfamily SymbolExpression} somewhere in the tree.
    
    \item {\ttfamily substitute(map)} takes a table mapping expressions to other expressions and recursively maps each instance of an expression with its corresponding table expression.
    
    \item {\ttfamily expand()} expands an expression, turning sums of products into products of sums.
    
    \item {\ttfamily factor()} factors an expression, turning products of sums into sums of products.
    
    \item {\ttfamily isatomic()} determines whether an expression is \emph{atomic}, i.e., whether it has any subexpression fields. Otherwise, an expression is \emph{compound}. Atomic expressions and compound expressions are both sub-interfaces to expressions, and all classes inherit from one of those interfaces.
    
    \item {\ttfamily isconstant()} determines whether an expression is atomic and contains no variables. See Section B.2 for more details.
    
    \item {\ttfamily isrealconstant()} determines whether an expression is a real number in the mathematical sense, such as $2$, $\sqrt{5}$, or $\sin(3)$.
    
    \item {\ttfamily iscomplexconstant()} determines whether an expression is a complex number in the mathematical sense, such as $3 + \sqrt{2}i$.
    
    \item {\ttfamily tolatex()} converts an expression to \LaTeX{} code.
\end{itemize}

The number of core methods should generally be kept small, since every new type of expression must implement all of these methods. The exception to this, of course, is core methods that call other core methods that provide a unified interface to expressions. For instance, {\ttfamily size()} calls {\ttfamily subexpressions()}, so it only needs to be implemented in the expression interface.

All expressions should also implement the {\ttfamily \_\_tostring} and {\ttfamily \_\_eq} metamethods. Unlike regular methods, metamethods cannot be inherited using Lua, thus every expression object created by a constructor must assign a metatable to that object.

\begin{itemize}
    \item {\ttfamily \_\_tostring} provides a human-readable version of an expression for printing within Lua. Used for debugging purposes.
    
    \item {\ttfamily \_\_eq} determines whether an expression is structurally identical to another expression.
\end{itemize}

\subsection{Algebra Package}

The algebra package contains functionality for arbitrary-precision arithmetic, polynomial arithmetic and factoring, symbolic root finding, and logarithm and trigonometric expression classes. It requires the core package to be loaded. There are a couple of features of note in this package.

\subsection{Calculus Package}

The calculus package contains expression types for integration and differentiation. It requires both the algebra and core packages to be loaded. There isn't anything special going on in this package from a design perspective, so we don't have much to say for this section.

\subsection{Constructing Expressions \& Basic Parsing}

Algebraic computations on the Lua end typically involve constructing the desired expression and applying one or more core operations to the expression. If a non-core operation is applied, then this core expression will be {\ttfamily autosimplify()}.

Naively, one could nest class constructors to build an expression. For instance, the Lua code for evaluating the integral \[ \int_0^6 \sin(x) + 2x^2 + 3 \hspace{2pt}\mathrm{d}x\] and storing the result in a variable {\ttfamily r} is as follows:

\begin{minted}{lua}
r = IntegralExpression(BinaryOperation(BinaryOperation.ADD,
            {TrigExpression("sin", SymbolExpression("x")),
            BinaryOperation(BinaryOperation.MUL,
                {Integer(2),
                BinaryOperation(BinaryOperation.POW,
                {SymbolExpression("x"), Integer(2)})}),
            Integer(3)}),
        SymbolExpression("x"),
        Integer(0),
        Integer(6)):autosimplify()
\end{minted}

This is more than unwieldy, so a number of constructs are added to make using and developing the CAS easier. Metatables for many binary operations allow natural symbols to act as constructors that use Lua's build-in operator precedence:

\begin{minted}{lua}
r = IntegralExpression(TrigExpression("sin", SymbolExpression("x")) +
        Integer(2) * SymbolExpression("x") ^ Integer(2) +
        Integer(3),
        SymbolExpression("x"),
        Integer(0),
        Integer(6)):autosimplify()
\end{minted}

Several functions are added as shorthand for constructors that mirror familiar syntax. These function names are in all caps to reduce the risk of collisions.

\begin{minted}{lua}
r = INT(SIN(SymbolExpression("x")) + Integer(2) * SymbolExpression("x") ^ Integer(2) 
    + Integer(3),
        SymbolExpression("x"),
        Integer(0),
        Integer(6)):autosimplify()
\end{minted}

This is still fairly unwieldy, mostly because every string and number need to be wrapped in a constructor to convert them to expressions. Unfortunately, this syntax is also the most awkward to reduce in Lua, and the inner workings of the code just uses the longer constructors.

For \LaTeX{} users, or for Lua users who are looking for a simpler paradigm, a parser was implemented that takes the code input as a string and wraps numbers in the {\ttfamily Integer} constructor, then evaluates the code using Lua's {\ttfamily load()} function. The {\ttfamily vars()} command also allows variables to be declared as symbol expressions with the same name, at the cost of being able to use these symbols to reference other objects. Using the parser, the code now looks like:

\begin{minted}{lua}
CASparse([[
    vars("x")
    r = int(sin(x) + 2*x^2 + 3, x, 0, 6)
]])
\end{minted}


\end{document}